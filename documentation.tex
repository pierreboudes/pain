\documentclass{article}
\usepackage[scale=0.74,vmargin=1.4cm ]{geometry}
\usepackage[utf8]{inputenc}
\usepackage[T1]{fontenc}
\usepackage[francais]{babel}
\usepackage{ifpdf}
\ifpdf
\usepackage[pdftex]{graphicx}
\else
\usepackage[dvips]{graphicx}
\usepackage{pstricks, pst-tree, pst-node}
\fi
\usepackage{calc}
\usepackage{moreverb}
\usepackage{url}
\usepackage{multicol}
\usepackage{tikz}
\usetikzlibrary{shapes,arrows}
\usepackage{rotating}
\usepackage{subfigure}

\begin{document}
\title{Documentation pain}
\author{Pierre Boudes}

\section{Tables de la base de donnée}

\begin{itemize}
\item cours : description du cours du point de vue étudiant (comme dans la maquette et le contrôle des connaissance). Formation, année, semestre et parfum (mention, spécialité, etc.). Volume horaire du cours magistral, TD, TP, soutien (ou prérentrée), contrôle et/ou encadrement  (comme s'il n'y avait qu'un seul groupe et sans appliquer de calcul d'un équivalent TD). Petit texte pour précisions éventuelles. Pour plus tard : code geisha. Nombre de crédits. Responsable.
\begin{verbatim}
CREATE TABLE `cours` (
  `id_cours` mediumint(8) unsigned NOT NULL AUTO_INCREMENT,
  `id_formation` mediumint(8) unsigned NOT NULL,
  `semestre` tinyint(3) unsigned NOT NULL,
  `nom_cours` varchar(80) COLLATE utf8_swedish_ci NOT NULL,
  `credits` tinyint(3) unsigned DEFAULT NULL,
  `id_enseignant` mediumint(8) unsigned NOT NULL,
  `cm` double unsigned DEFAULT NULL,
  `td` double unsigned DEFAULT NULL,
  `tp` double unsigned DEFAULT NULL,
  `alt` double unsigned DEFAULT NULL,
  `descriptif` text COLLATE utf8_swedish_ci,
  `code_geisha` varchar(16) COLLATE utf8_swedish_ci DEFAULT NULL,
  `modification` timestamp NOT NULL DEFAULT CURRENT_TIMESTAMP,
  PRIMARY KEY (`id_cours`)
) DEFAULT CHARSET=utf8 COLLATE=utf8_swedish_ci;
\end{verbatim}
\item enseignants. Statut, s'il ne s'agit pas de vacations, volume équivalent TD du service de base (les décharges, CRCT et autres sont à compter comme des tranches de cours, donc le service de base est normalement de 192H pour les tous les titulaires et ATER temps plein). 
\begin{verbatim}
CREATE TABLE `enseignant` (
  `id_enseignant` mediumint(8) unsigned NOT NULL AUTO_INCREMENT,
  `prenom` varchar(40) COLLATE utf8_swedish_ci NOT NULL,
  `nom` varchar(40) COLLATE utf8_swedish_ci NOT NULL,
  `statut` varchar(40) COLLATE utf8_swedish_ci DEFAULT NULL,
  `email` varchar(40) COLLATE utf8_swedish_ci DEFAULT NULL,
  `telephone` varchar(20) COLLATE utf8_swedish_ci DEFAULT NULL,
  `bureau` varchar(40) COLLATE utf8_swedish_ci DEFAULT NULL,
  `service` smallint(5) unsigned DEFAULT '192',
  `modification` timestamp NOT NULL DEFAULT CURRENT_TIMESTAMP,
  PRIMARY KEY (`id_enseignant`)
) DEFAULT CHARSET=utf8 COLLATE=utf8_swedish_ci;

\end{verbatim}


\item tranches (ou interventions). Portion de cours attribué à un enseignant ou à atrribuer (prétranche), éventuellement annotée par un identificateur de groupe. Volume horaire devant les étudiants pour chaque type d'intervention (CM, TD, TP, soutien, contrôle, encadrement)  et volume en équivalent TD avec remarque éventuelle sur la méthode de conversion si ce volume est entré manuellement. On peut entrer plusieurs tranches pour un même enseignant et un même cours (si ça fait sens). Voir s'il faut fusionner automatiquement les tranches relâchées (redevenues prétranches). 
\begin{verbatim}
CREATE TABLE `tranche` (
  `id_tranche` mediumint(8) unsigned NOT NULL AUTO_INCREMENT,
  `id_cours` smallint(5) unsigned NOT NULL,
  `id_enseignant` mediumint(5) DEFAULT '-1',
  `type` tinyint(3) unsigned NOT NULL DEFAULT '1',
  `groupe` tinyint(3) unsigned DEFAULT '0',
  `cm` double unsigned DEFAULT NULL,
  `td` double unsigned DEFAULT NULL,
  `tp` double unsigned DEFAULT NULL,
  `alt` double unsigned DEFAULT NULL,
  `type_conversion` tinyint(3) unsigned DEFAULT NULL,
  `remarque` text COLLATE utf8_swedish_ci,
  `htd` double unsigned DEFAULT NULL,
  `descriptif` text COLLATE utf8_swedish_ci,
  `modification` timestamp NOT NULL DEFAULT CURRENT_TIMESTAMP,
  PRIMARY KEY (`id_tranche`)
) DEFAULT CHARSET=utf8 COLLATE=utf8_swedish_ci;
\end{verbatim}
\end{itemize}

Tables complémentaires :
\begin{itemize}
\item formations par années et parfums (chaînes de caractère). Pour aider à la création des cours (normaliser les intitulés des formations et parfums). Numéro d'ordre pour l'affichage. Responsable de cette année pour ce parfum.

\begin{verbatim}
CREATE TABLE `formation` (
  `id_formation` mediumint(8) unsigned NOT NULL AUTO_INCREMENT,
  `numero` smallint(5) unsigned NOT NULL,
  `nom` varchar(40) COLLATE utf8_swedish_ci NOT NULL,
  `annee_etude` tinyint(3) unsigned NOT NULL,
  `parfum` varchar(40) COLLATE utf8_swedish_ci DEFAULT NULL,
  `annee_universitaire` year(4) DEFAULT '2009',
  `id_enseignant` smallint(5) unsigned DEFAULT NULL,
  `modification` timestamp NOT NULL DEFAULT CURRENT_TIMESTAMP,
  PRIMARY KEY (`id_formation`)
) DEFAULT CHARSET=utf8 COLLATE=utf8_swedish_ci;

\end{verbatim}
\end{itemize}



\section{Rôles}

\begin{itemize}
\item Département : toute modification. En particulier création des enseignants et des tranches de réduction de service (décharges, modulations, forfait horaire de responsabilité).
\item Responsable de formation : créer/supprimer des cours et des prétranches.
\item Responsable d'année : créer/supprimer des tranches sans enseignant pour chaque cours, renommer des groupes déjà alloués, modifier des cours.
\item Enseignant : sélectionner des prétranches en entier ou par portion pour s'en payer une tranche, afficher n'importe quel récapitulatif.
\end{itemize}

\section{Interface}
\begin{itemize}
\item Récapitulatif global : les cours par formation et par année, avec notamment le nombre d'heures pourvues et restant à pourvoir pour l'année en cours (somme sur les tranches) pour chaque cours, pour chaque année, pour chaque formation, et globalement.
\item Vue sur un cours, pour manipulation des tranches.
\item Feuille de service pour un enseignant.
\item Vue des services de l'ensemble des enseignants non extérieurs/vacataires (liste).
\item Intervenants extérieurs par formation avec volumes (pour plus tard).
\item Réductions de service par type avec liste des bénéficiaires (pour plus tard).
\item Liste des intervenants d'une formation type annuaire, par semestre ou année, avec ou sans parfum (pour plus tard).
\item Contrôle des connaissances (pour plus tard).
\end{itemize}


\section{Fonctions}



\end{document}
